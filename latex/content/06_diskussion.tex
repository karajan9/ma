\chapter{Zusammenfassung und Ausblick}

Ein Ziel dieser Arbeit war es, im Temperaturbereich um und unter der Glasübergangstemperatur von $T_g = \SI{333}{K}$ von CRN nach Dynamik zu suchen. Dafür wurden sowohl stimulierte Echos als auch die Breiten pulslängenahängiger Spektren aufgenommen und analysiert. Es wurde erwartet, dass die Zeitskalen der gefundenen Zeitkonstanten die Existenz eine Betaprozesses, wie sie Abbildung \ref{fig:einl:zuernpaper} andeutet, unterstützen. Dies konnte jedoch nicht bestätigt werden -- die gefunden Zeitkonstanten liegen mehrere Größenordnungen über den Erwartungen. Eine mögliche Erklärung wäre, dass die Stoffe CRN und CKN in diesem Bereich unterschiedliche Eigenschaften aufweisen und der Betaprozesses tatsächlich nur für CKN erwartet werden kann. Dies ist aber aufgrund der Ähnlichkeit der Stoffe in anderen Eigenschaften eher unwahrscheinlich. Eine Klärung könnte durch Untersuchungen mit anderen spektroskopischen Methoden versucht werden.

Desweiteren wurde eine Untersuchung der Änderung Linienform von CRN-Spektren im Verlauf der Temperatur durchgeführt. Dazu wurden sowohl experimentelle Spektren an zwei Spektrometern mit unterschiedlicher Larmorfrequenz aufgenommen, als auch Simulationen auf Basis eines isotropen Sprungs mit einer Czjzek-Verteilung als Ausgangspunkt durchgeführt. Im Temperaturbereich unter $\SI{375}{K}$ weisen Simulation und Experimente eine gute Übereinstimmung der Linienform auf; zu höheren Temperaturen sind starke Abweichungen zu beobachten. Hier sind weitere Untersuchungen -- und möglicherweise eine Anpassung des verwendeten Bewegungsmodells -- notwendig, um über den ganzen Temperaturbereich eine zufriedenstellende Übereinstimmung zu erreichen.

Zu dem hier verwendeten Czjzek-Modell mit isotropem Sprung wurde ergänzend ein Oktaeder-Modell vorgeschlagen. Dieses kann bestimmte, in diesem Kontext erwünschte, Eigenschaften wie die Czjzek-Linienform replizieren, erlaubt jedoch deutlich feinere Anpassungen des Bewegungsmodells. Während die Anfänge dieses Modells vielversprechend scheinen, ist mehr Arbeit notwendig, um es zum Einsatz zu bringen und daraus Erkenntnisse zu gewinnen.

Die experimentellen Spektren sind im Bezug auf die unterschiedlichen verwendeten Larmorfrequenzen in sich konsistent, zeigen jedoch sowohl bei der Halbwertsbreite als auch bei den Schwerpunkten der Spektren leichte Unterschiede. Messungen an weiteren Spektrometern mit zusätzlichen Larmorfrequenzen könnten helfen, die Einflüsse von chemischer Verschiebung und Quadrupol-Wechselwirkung zweiter Ordnung zu trennen und aufzuschlüsseln. 

Die experimentellen Spektren wurden zudem mit der Theorie der Quadrupol-Wech\-sel\-wir\-kung zweiter Ordnung verglichen. Dazu wurden die experimentell gemessenen Werte für $T_1$ sowie die Halbwertsbreiten und Schwerpunkte von Spektren mit den von der Theorie vorhergesagten verglichen. Dabei wurden für unterschiedliche verwendete Spektraldichten unterschiedliche Übereinstimmungen gefunden. Die Spektraldichte $J_\text{BPP}$ zeigt für Temperaturen über $\SI{375}{K}$ gute Deckungen mit der Halbwertsbreite und den Schwerpunkten der experimentellen Spektren, aber deutliche Abweichungen bei $T_1$. Die Spektraldichten $J_\text{CC}$ und $J_\text{CD}$ zeigten das umgekehrte Verhalten: Während für Temperaturen über $\SI{350}{K}$ besonders mit $J_\text{CC}$ eine gute Übereinstimmung zu $T_1$ erreicht werden konnte, gibt es hier für die Halbwertsbreiten und Schwerpunkte größere Abweichungen. Keine der verwendeten Spektraldichten konnte eine Deckung aller drei untersuchten Eigenschaften mit einem Parametersatz erreichen. Für die Gründe hierfür sind weitere Untersuchungen notwendig. Messungen zu höheren Temperaturen, beispielsweise mit dem Hochtemperatur-Probenkopf, könnten klären, ob die Experimente auch in anderen als den hier untersuchen Temperaturbereichen konsistent mit der Theorie sind.

Der Einfluss eines kurzen $T_1$ bei Temperaturen über $\SI{370}{K}$ konnte in der Halbwertsbreite beobachtet, und, für einen besseren Vergleich mit der Theorie, rechnerisch eliminiert werden.

