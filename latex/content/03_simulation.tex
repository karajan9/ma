\chapter{Simulation}\label{chapter:simulation}

Simulationen werden in vielen Disziplinen der Physik gerne verwendet, da sie es ermöglichen, Sachverhalte schnell, flexibel und kostengünstig zu untersuchen. Weiter noch können die Grenzen des experimentell Realisierbaren überschritten werden, da diese Grenzen in der simulierten Umgebung nicht zwingend existieren müssen.

Auf der Gegenseite müssen sich Simulationen, die ein Experiment wie die NMR-Spektroskopie nachstellen, meist auf eine Auswahl der unter Laborbedingungen vorliegenden Einflüsse und Wechselwirkungen beschränken. Zu groß ist die Zahl der (teilweise auch unbekannten) Parameter. Daher kann eine Simulation in der Regel nur als eine Näherung eines Experiments verstanden werden.

Daher ist der Vergleich von Simulation und Experiment von besonderer Bedeutung. Soll zum Beispiel untersucht werden, ob eine bestimmte gemessene Eigenschaft durch eine bestimmte Wechselwirkung hervorgerufen wird, kann eine entsprechende Simulation durchgeführt werden, wobei bei dieser die Zahl der möglichen Einflüsse deutlich reduziert werden kann -- andere Wechselwirkungen als die von Interesse müssen nicht simuliert werden. Ein Vergleich der simulierten Eigenschaften den gemessenen kann dann einen bestehenden Verdacht bestärken oder entkräften.


Aus diesen Gründen wurde an diesem Lehrstuhl eine Simulationssoftware von J. Beerwerth in der Programmiersprache C++ geschrieben und aktiv weiterentwickelt. Dabei handelt es sich um eine Random-Walk-Simulation. Diese soll NMR-Experimente simulieren und beinhaltet eine Vielzahl von Möglichkeiten. Während nicht jeder Teil der Simulationssoftware explizit benannt werden soll, soll dennoch der Aufbau nachvollzogen werden, um eine Einschätzung ihrer Möglichkeiten bieten zu können. Jede Simulation besteht aus drei Komponenten, die unabhängig von einander gewählt werden können: Die Art der Messung (beispielsweise ein 2D-Spektrum, Hahn-Echo oder ein stimuliertes Echo), das Bewegungsmodell (zum Beispiel Sprünge zwischen $N$ festen Plätzen oder ein isotroper Sprung) und die zu berücksichtigen Wechselwirkungen (wie die Quadrupol-Wechselwirkung erster oder zweiter Ordnung, oder die chemische Verschiebung).

Die Software ist modular aufgebaut, um Erweiterungen, beispielsweise einer anderen Wechselwirkung, leicht zuzulassen. Dazu werden die drei Komponenten, die Wechselwirkung, das Bewegungsmodell, und die Art der Messung, mit Hilfe von abstrakten Klassen implementiert. Diese lauten \texttt{Frequency}, \texttt{MotionalModel}, und \texttt{Measurement}. Für zusätzliche Wechselwirkung muss nur von der entsprechende abstrakte Klasse \texttt{Frequency} geerbt werden um sie anstelle einer anderen Wechselwirkung verwendet zu können.

Die eigentliche Arbeitsweise der Software soll an den drei Komponenten verdeutlicht werden, die für diese verwendet wurden: Die FID-Pulsfolge, die Quadrupol-Wechselwirkung zweiter Ordnung, und das Bewegungsmodell des isotropen Sprungs.







Es können verschiedene Experiment wie Hahn-Echos, 2D-Spektren oder stimulierte Echos simuliert werden, 







Die kann, aufgrund der fehlenden Übereinstimmung 







Trajektorien







Zunächst muss entschieden werden, welche Art von Messung simuliert werden soll. Dies schließt zum Beispiel bestimmte Pulsfolgen wie FIDs oder Hahn-Echos ein, aber auch die Möglichkeit Spektren zu erstellen, ohne vollständige Trajektorien berechnen zu müssen.


Die Klasse \texttt{NMRSimulation} bildet das Herzstück der Software. Per Komandozeile übergebene Parameter werden geparst und gespeichert, oder an die richtigen Stellen weitergeleitet.


Die benötigten Zufallszahlen hierzu werden von einem Mersenne-Twister Algorithmus bereitgestellt, der heutzutage einen Standard in vielen Programmiersprachen darstellt und gute statistische Eigenschaften aufweist.



Das Czjzek-Modell nach \cite{} geht von *** aus. Daraus ergeben sich bestimmte Wahrscheinlichkeitsverteilungen für die Parameter $V_{zz}$ und $\eta$, die, wie in Kapitel *** erwähnt, zur Parametrisierung eines EFGs genügen. Aus dieser Verteilung kann nun ein Parametersatz gezogen werden, der den simulierten EFG darstellen soll. Dies geschieht mit einem Neumannschen Rückweisungsverfahren

Die Verteilung *** gibt die Wahrscheinlichkeit für die Kombination aus $V_{zz}$ und $\eta$ an

