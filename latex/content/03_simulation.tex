\chapter{Simulation}\label{chapter:simulation}

Die hier verwendete Simulationssoftware, geschrieben in C++, wurde an diesem Lehrstuhl von J. Beerwerth entwickelt.


Simulationen werden gerne verwendet um schnell und kostengünstig


Ein besonderer Wert von Simulationen offenbaren sich im Vergleich mit experimentellen Daten.



Während nicht jeder Teil der Simulationssoftware explizit benannt werden soll, soll dennoch der Aufbau nachvollzogen werden, um eine Einschätzung ihrer (Erweiterungs-)Möglichkeiten bieten zu können.


Trajektorien


Jede Simulation besteht aus drei Komponenten, die unabhängig von einander gewählt werden können: Die Art der Messung (beispielsweise ein 2D-Spektrum oder ein Hahn-Echo), das Bewegungsmodell (zum Beispiel Sprünge zwischen $N$ festen Plätzen oder ein isotroper Sprung) und die zu berücksichtigen Wechselwirkungen (wie die Quadrupol-Wechselwirkung oder die chemische Verschiebung).




Zunächst muss entschieden werden, welche Art von Messung simuliert werden soll. Dies schließt zum Beispiel bestimmte Pulsfolgen wie FIDs oder Hahn-Echos ein, aber auch die Möglichkeit Spektren zu erstellen, ohne vollständige Trajektorien berechnen zu müssen.


Die Klasse \texttt{NMRSimulation} bildet das Herzstück der Software. Per Komandozeile übergebene Parameter werden geparst und gespeichert, oder an die richtigen Stellen weitergeleitet.


Die benötigten Zufallszahlen hierzu werden von einem Mersenne-Twister Algorithmus bereitgestellt, der heutzutage einen Standard in vielen Programmiersprachen darstellt und gute statistische Eigenschaften aufweist.



Das Czjzek-Modell nach \cite{} geht von *** aus. Daraus ergeben sich bestimmte Wahrscheinlichkeitsverteilungen für die Parameter $V_{zz}$ und $\eta$, die, wie in Kapitel *** erwähnt, zur Parametrisierung eines EFGs genügen. Aus dieser Verteilung kann nun ein Parametersatz gezogen werden, der den simulierten EFG darstellen soll. Dies geschieht mit einem Neumannschen Rückweisungsverfahren

Die Verteilung *** gibt die Wahrscheinlichkeit für die Kombination aus $V_{zz}$ und $\eta$ an

