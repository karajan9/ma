\chapter{Theoretische Einführung}
\label{chapter:theo}

\section{Theoretische Grundlagen der NMR} \label{section:theo:grundlagen}


Ziel der NMR ist es, Kernspins einer Probe zu manipulieren und zu beobachten.
Da Spins $I$ über
\begin{align}
    \vec{\mu} = \gamma \hbar \vec{I}
\end{align}
mit magnetischen Dipolen $\mu$ verknüpft sind, ist die Beeinflussung von Spins mithilfe von Magnetfeldern möglich. Der Proportionalitätsfaktor $\gamma$ ist als gyromagnetisches Verhältnis bekannt und eine isotopenabhängige Konstante.

Werden die Kernspins einem Magnetfeld $\vec{B}_0 = (0, 0, B_{0,z})$ ausgesetzt, spalten die Energiezustände nach Zeeman auf. Die potentielle Energie eines magnetischen Dipols ist $E = - \vec{\mu} \vec{B}_0$. Mit obigen Beziehungen resultiert demnach $E = - \gamma \hbar I_z B_{0,z}$. Die Eigenwerte von $I_z$ lauten $m = -I, -I + 1, \dots, I - 1, I$; somit ist die Energiedifferenz zwischen zwei benachbarten Zuständen $\Delta E = E_{m+1} - E_{m} = - \gamma \hbar B_{0,z}$. Die damit verbundene Kreisfrequenz
\begin{align}
    \omega_L = \frac{\Delta E}{\hbar} = - \gamma B_{0,z} \label{eqn:lamorfrequenz}
\end{align}
ist die Larmorfrequenz oder Resonanzfrequenz. Der Konvention folgend wird $\omega_L$ nachfolgend als positiv angenommen.



In Anlehnung an das klassische Pendant, findet sich folgende Formel für den Drehimpuls $I$ und die Magnetisierung $M = \sfrac{1}{V} \sum_i \mu_i = \sfrac{\gamma \hbar}{V} \sum_i I_i$:
\begin{align}
    \frac{\text{d}\vec{M}}{\text{d}t} = \omega_L \times \vec{M}. \label{eqn:drehmoment}
\end{align}
Diese Gleichung lässt sich durch:
\begin{align}
    M_x (t) &= M_x(0) \cos(\omega_L t - \phi) \\
    M_y (t) &= M_y(0) \sin(\omega_L t - \phi) \\
    M_z (t) &= M_z(0).
\end{align}
lösen; diese sind bekannt als Bloch-Gleichungen.

Dies bedeutet, dass aus der Gleichgewichtsorientierung $M_z(\infty) = M_\infty$ ausgelenkte magnetische Dipole nicht zu dieser zurück relaxieren, sondern eine Ausweichbewegung vollziehen und mit der Kreisfrequenz $\omega = \omega_L$ um die Achse des Magnetfeldes präzedieren.

Um die folgenden Überlegungen einfacher zu gestalten, wird nun von dem Laborsystem in ein rotierendes Koordinatensystem übergegangen. Die Drehung des Koordinatensystems soll der der Magnetisierung dabei dergestalt gleichen, dass diese im neuen Koordinatensystem stationär erscheint; dies bedeutet also eine Drehung des Koordinatensystems um die Magnetfeldachse mit einer Kreisfrequenz $\omega_L$.

Dies hat nach Formel \eqref{eqn:lamorfrequenz} zur Folge, dass das Magnetfeld $B_0$ im rotierenden Koordinatensystem als geschwächt erscheint: $B_\text{eff} = B_0 - \sfrac{\omega}{\gamma}$. Ist $\omega = \omega_L$, so ist das effektive Magnetfeld null.


Neben diesem Vorwissen ist es nötig, Spins aktiv beeinflussen zu können, um in der Lage zu sein effektiv Informationen zu sammeln. Daher wird häufig die Probe in einer Spule untergebracht, deren Achse senkrecht zu der des Magnetfeldes steht; die Konvention richtet diese Spule entlang der $x$-Achse aus. Wird an der Spule eine Wechselspannung angelegt, werden die Spins einem entsprechenden Magnetfeld $\vec{B}_1 = 2 B_{1,x} \cos(\omega_1 t)$ ausgesetzt.

Da die Spule im Laborsystem stationär ist, rotiert das entstehende Magnetfeld im rotierenden Koordinatensystem mit $-\omega$. Um dies zu vermeiden wird $B_1$ als Superposition zweier Drehungen dargestellt:
\begin{align}
    \vec{B}_1 = B_1 \left[ \myvec{\cos(\omega) \\ \sin(\omega) \\ 0} + 
                      \myvec{\cos(\omega) \\ -\sin(\omega) \\ 0} \right].
\end{align}
Im rotierenden Koordinatensystem ergibt sich dies nun zu
\begin{align}
    \vec{B}_1 = B_1 \left[ \myvec{1 \\ 0 \\ 0} + 
                      \myvec{\cos(2\omega) \\ -\sin(2\omega) \\ 0} \right],
\end{align}
was bedeutet, dass sich das Magnetfeld als eine Superposition aus einem stationären Teil und einem Teil, der mit $-2\omega$ rotiert, darstellen lässt. Letzter kann aufgrund der hohen Frequenz vernachlässigt werden -- dies ist als „rotating wave approximation“ bekannt --, sodass nur der stationäre Anteil im rotierenden Koordinatensystem verbleibt.

Wird das $B_1$-Feld vom Experimentierenden aktiviert, so vollführt die mittlere Magnetisierung nach Formel \eqref{eqn:drehmoment} eine Präzessionsbewegung senkrecht zu diesem Magnetfeld. Mit einem Puls einer bestimmten Länge
\begin{align}
	t_\text{Puls} = \frac{\phi}{- \gamma B_1} \label{eqn:theo:pulslaenge}
\end{align}
lässt sich die Magnetisierung manipulieren, so kann beispielsweise $M_\infty$ mit einem $\phi = \SI{90}{\degree}$-Puls in die $x$-$y$-Ebene, auch Detektionsebene genannt, geklappt, oder mit einem 180$^\circ$-Puls invertiert werden. Durch das Kombinieren von Pulsen lassen sich Pulsfolgen entwickeln, um interessante Eigenschaften einer Probe zu untersuchen. Einige dieser werden in Kapitel \ref{section:theo:pulsfolgen} vorgestellt.



\section{Theoretische Aspekte der Relaxation} \label{section:theo:relax}

In allen realen Systemen gibt es durch unterschiedliche Wechselwirkungen Relaxationen der longitudinalen und transversalen Magnetisierung, welche ein essentieller Teil jeder NMR-Messung sind und im Folgenden vorgestellt werden sollen.

Werden Kernspins, beispielsweise durch einen 90$^\circ$-Puls, aus ihrem Gleichgewichtszustand ausgelenkt, sind sie nicht mehr im Boltzmann-Gleichgewicht und sollten daher in den Gleichgewichtszustand relaxieren. Bei den für die hier verwendeten Temperaturen von mehreren Hundert Kelvin sind jedoch die Besetzungsverhältnisse $\sfrac{N_+}{N_-} = \exp (\sfrac{- \hbar \gamma B}{k_B T})$ fast beliebig nah an 1 und die thermisch induzierten Relaxationsraten daher nahezu verschwindend gering. Energie aus anderen Quellen kann jedoch Übergänge zwischen den Zuständen und damit Relaxation ermöglichen.

Durch die zufälligen Bewegungen der Atome oder Moleküle einer Probe ändern sich für ein untersuchtes Atom die entsprechenden lokalen elektromagnetischen Felder. Enthält das Spektrum der Feldfluktuationen Frequenzen der Larmorfrequenz (oder, je nach Wechselwirkung, Vielfache davon), können Relaxationsprozesse auftreten und somit schlussendlich die gesamte Magnetisierung in den Gleichgewichtszustand zurückkehren. Es ist offensichtlich, dass die Spektraldichte $J(\omega)$, die die Häufigkeit der entsprechenden Frequenzen angibt, eine wichtige Rolle bei der Beschreibung der Relaxation spielt. Zu berücksichtigende Wechselwirkungen sind -- je nach untersuchtem Stoff bzw. Kern -- zum Beispiel die chemische Verschiebung, die Dipol-Dipol-Wechselwirkung oder die Quadrupol-Wechselwirkung, welche in Kapitel \ref{section:theo:qww} vorgestellt wird.

Die durch die Bewegung der Atome induzierte Relaxation wird longitudinale Relaxation oder Spin-Gitter-Relaxation genannt; die zugehörige Konstante $T_1$ gibt an, auf welcher Zeitskala sie stattfindet.

Die transversale Relaxation oder Spin-Spin-Relaxation basiert, wie der Name suggeriert, auf Wechselwirkungen zwischen mehreren Spins. Die Stärke der Dipol-Dipol-Wechselwirkung ist für jeden Spin aufgrund der leicht verschiedenen Positionen und Winkeln der Spins zueinander ebenfalls leicht unterschiedlich sowie winkelabhängig. Das bedeutet, dass nicht alle Spins mit der gleichen Frequenz prädizieren, sondern sich mit der Zeit eine Phase bildet und somit die mittlere Magnetisierung reduziert wird. Die Größe dieser Relaxation wird durch $T_2$ beschrieben.


Während sich die Grundlagen des letzten Kapitels gut und verständlich mit klassischen Konzepten darstellen lassen, ist es für die theoretische Beschreibung der Relaxation unabdinglich, zu quantenmechanischen Rechnungen überzugehen. Die folgenden Rechnungen finden sich in detaillierterer Form in \cite[S. 191-195]{benesi} und \cite[S. 108-110]{spiess}.

Der Ausgangspunkt ist der Dichteoperator $\hat{\rho}(t)$, der interne Hamilton-Operator im rotierenden Koordinatensystem $\hat{H}_\text{rot}(t)$, und die Liouville-von-Neumann Gleichung
\begin{align}
    \frac{\text{d} \hat{\rho}_\text{rot}}{\text{d}t} = -i [\hat{H}_\text{rot}(t), \hat{\rho}_\text{rot}].
\end{align}

$\hat{H}_\text{rot}(t)$ ist aufgrund der erwähnten zufälligen Molekülbewegungen nicht konstant, weswegen die Gleichung bis zur zweiten Ordnung integriert werden muss. Wird davon die zeitliche Ableitung gebildet, ergibt sich mit der Substitution $t' = t - \tau$ folgende Gleichung:
\begin{align}
    \frac{\text{d} \hat{\rho}_\text{rot}}{\text{d}t} = - i [\hat{H}_\text{rot}(t), \hat{\rho}_\text{rot}(0)] - \int_0^t \text{d} \tau \left[\hat{H}_\text{rot}(t), [\hat{H}_\text{rot}(t-\tau), \hat{\rho}_\text{rot}(0)] \right].
\end{align}

Nach \cite[S. 276]{spiess} sind vier Annahmen notwendig, um schlussendlich zur Mastergleichung zu gelangen:
\begin{itemize}
\item $\hat{H}_\text{rot}(t)$ und $\hat{\rho}_\text{rot}(0)$ sind unkorreliert.
\item $\hat{\rho}_\text{rot}(0)$ kann durch $\hat{\rho}_\text{rot}(t)$ ersetzt werden.
\item Die Integration kann von $t$ auf $\infty$ ausgeweitet werden.
\item Höhere als die zweite Ordnung der vorherigen Gleichung können vernachlässigt werden.
\end{itemize}
Zudem soll $\hat{\rho}' = \hat{\rho}_\text{rot} - \hat{\rho}_\text{eq}$ gelten, sodass $\hat{\rho}'$ die Abweichungen vom Gleichgewichtszustand $\hat{\rho}_\text{eq}$ darstellt.
Damit ergibt sich:
\begin{align}
    \frac{\text{d} \hat{\rho}'}{\text{d}t} = - \int_0^t \text{d} \tau \left< \left[\hat{H}_\text{rot}(t), [\hat{H}_\text{rot}(t-\tau), \hat{\rho}'(t)] \right] \right>. \label{eqn:mastergleichung}
\end{align}

Das Lösen dieser Gleichung gestaltet sich recht umfangreich, daher soll die Rechnung hier nur in den groben Zügen nachvollzogen werden.
Für die Darstellung des Hamilton-Operators werden sphärische Tensoren $\hat{T}^{l,m}$ verwendet, sodass sich mit den Laborsystem-Tensoren $A^{l,m}(t)$, eingesetzt in die Mastergleichung \eqref{eqn:mastergleichung}, Folgendes ergibt:
\begin{align}
\begin{split}
    \frac{\text{d} \hat{\rho}'}{\text{d}t} =& -\sum_{l_1 = 0}^2 \sum_{l_2 = 0}^2 \sum_{m_1 = -l}^l \sum_{m_2 = -l}^l (-1)^{m_1+m_2} e^{i(m_1+m_2) \omega_0 t} \left[\hat{T}_{l_1,m_1}, [\hat{T}_{l_2,m_2}, \hat{\rho}'(0)] \right] \\ &\times \int_0^\infty \left< A_{l_1,m_1}(t) A_{l_2,m_2}(t-\tau) \right> e^{i m_2 \omega_0 \tau} \text{d} \tau
\end{split}
\end{align}

Da der Hamiltonoperator der Wechselwirkung $\hat{H}_\text{rot}(t)$ klein ist (gegenüber der Zeeman-Wechselwirkung), ist auch $\sfrac{\text{d} \hat{\rho}'}{\text{d}t}$ klein. Durch diese langsame Änderung mit der Zeit mitteln sich bei dem schnell variierenden $e^{i(m_1+m_2) \omega_0 t}$ alle Elemente außer bei $m_1 = - m_2$ heraus. Gleichzeitig muss, wegen der Wigner-Orthogonalität der Operatoren, $l_1 = l_2$ gelten.

Es ergibt sich eine Autokorrelationsfunktion von $A$; das Integral über diese kann nach einigen Umformungen als Spektraldichte $J(\omega)$ identifiziert werden. Schlussendlich kann damit die Mastergleichung wie folgt formuliert werden:
\begin{align}
    \frac{\text{d} \hat{\rho}'}{\text{d}t} = - C_\gamma \sum_{l = 0}^2 \sum_{m = -l}^l (-1)^{m+l} \left[\hat{T}_{l,m}, [\hat{T}_{l,-m}, \hat{\rho}'(0)] \right] \times J_{l,m}(m \omega_0).
\end{align}
$C_\gamma$ ist dabei eine vom Hamiltonoperator abhängige Konstante.

Wird diese Gleichung unter Ausnutzung von Kommutatorrelationen gelöst, ergeben sich folgende Gleichungen, die zur longitudinalen bzw. transversalen Relaxation korrespondieren:
\begin{align}
    \Braket{\frac{\text{d} \hat{I}_z}{\text{d}t}} =& - C_\gamma \sum_{l = 0}^2 \sum_{m = -l}^l (-1)^{l+m} \left[\hat{T}_{l,m}, \hat{T}_{l,-m}\right] \times m J_{l,m}(m \omega_0) \label{eqn:long_relax_theorie} \\
\begin{split}
    \Braket{\frac{\text{d} \hat{I}_\pm}{\text{d}t}} =& - C_\gamma \sum_{l = 0}^2 \sum_{m = -l}^l (-1)^{l+m} \left[\hat{T}_{l,m \pm 1}, \hat{T}_{l,-m}\right] \\ &\times \sqrt{l(l+1) - m(m \pm 1)} \times m J_{l,m}(m \omega_0) \label{eqn:trans_relax_theorie}
\end{split}
\end{align}


% \begin{align}
% \begin{split}
%     \Braket{\frac{\text{d} \hat{I}_z}{\text{d}t}} =& -\sqrt{2} C_\gamma \left( J(-\omega) + J(\omega) \right) \Braket{\hat{I}^{1,0}} \\
%     & - \sqrt{\frac{2}{5}} C_\gamma \left( 4J(-2\omega) + 4J(2\omega) + J(-\omega) + J(\omega) \right) \Braket{\hat{I}^{1,0}} \\
%     & - \sqrt{\frac{2}{5}} C_\gamma \left( 2J(-2\omega) + 2J(2\omega) - 2J(-\omega) - 2J(\omega) \right) \Braket{\hat{I}^{3,0}} \label{eqn:long_relax}
% \end{split}
% \end{align}
% \begin{align}
% \begin{split}
%     \Braket{\frac{\text{d} \hat{I}^{\pm}}{\text{d}t}} =& -C_\gamma (2J(0) + 2J(\mp \omega)) \Braket{\hat{I}^{1,\pm 1}} \\
%     & - \sqrt{\frac{2}{5}} C_\gamma (2J(\mp 2\omega) + 3J(\mp \omega) + 3J(0) + 2J(\pm \omega)) \Braket{\hat{I}^{1,\pm 1}} \\
%     & - \sqrt{\frac{2}{5}} C_\gamma (\sqrt{6}J(\mp 2\omega) - \sqrt{6}J(\mp \omega) - \sqrt{6}J(0) + \sqrt{6}J(\pm \omega)) \Braket{\hat{I}^{3,\pm 1}} \label{eqn:trans_relax}
% \end{split}
% \end{align}
Verschiedene Wechselwirkungs-Hamiltonoperatoren haben unterschiedliche Lösungen für die Kommutatorrelationen in Gleichungen \eqref{eqn:long_relax_theorie} und \eqref{eqn:trans_relax_theorie}, weswegen sich für Wechselwirkungen spezifische Relaxationen ergeben.





\section{Quadrupol-Wechselwirkung} \label{section:theo:qww}

Auf die im Kontext dieser Arbeit wichtigste Wechselwirkung soll nun eingegangen werden. Als die Quadrupol-Wechselwirkung wird die Wechselwirkung des Kernquadrupolmoment mit dem lokalen elektronischen Feld bezeichnet. Bei einem Kern mit $I = \sfrac{1}{2}$ ist dessen Ladung symmetrisch verteilt ist und daher kein Moment durch ein äußeres Feld zu beobachten. Diese Wechselwirkung tritt also nur bei Kernen mit $I \ge 1$ auf, ist dann aber meist dominant -- so auch in dem hier untersuchten Stoff CRN.

Der Hamiltonoperator der Quadrupol-Wechselwirkung lautet \cite[S. 208]{levitt}
\begin{align}
	\hat{H}_Q = \frac{eQ}{2I (2I - 1) \hbar} \hat{I} V \hat{I}.
\end{align}
Dabei ist $Q$ das Kernquadrupolmoment, $I$ die Spinquantenzahl und $V$ der elektronischen Feldgradienten (kurz EFG), der den Einfluss des äußeren Feldes, entstehend durch die Elektronenverteilung benachbarter Kerne, beschreibt.

Die Matrixelemente des EFG lassen sich aus dem elektrostatischen Potential $V$ wie folgt berechnen:
\begin{align}
	V_{ij} = \frac{\partial^2 V}{\partial x_i \partial x_j}.
\end{align}
Da die Reihenfolge der Ableitungen in einem konservativen Potential keine Rolle spielt, gilt $V_{ij} = V_{ji}$ -- der Tensor ist symmetrisch. Zudem gibt die Laplace-Gleichung $\nabla^2 V(r) = 0$, was dazu führt, dass der Tensor zudem spurlos ist. Dies lässt dem EFG 5 unabhängige Parameter.

Wird der EFG auf Hauptachsenform gebracht, ergeben sich drei Eigenwerte. Der Konvention folgend werden diese wie folgt sortiert: $\lvert V_{zz} \rvert \ge \lvert V_{yy} \rvert \ge \lvert V_{xx} \rvert$. Da der EFG, wie erwähnt, spurlos ist, lassen sich diese Informationen in nur zwei Variablen darstellen:
\begin{align}
	eq                & = V_{zz}                         \\
	\text{und}\; \eta & = \frac{V_{xx} - V_{yy}}{V_{zz}}
\end{align}
$\eta$ wird Asymmetrieparameter genannt und kann Werte zwischen $\SI{0}{}$ und $\SI{1}{}$ annehmen. Er kann, genauso wie $eq$ darüber Auskunft geben, wie sich die Form des EFG visualisieren lässt: $\eta = 0$ gibt beispielsweise einen rotationssymmetrischen EFG an. In diesem Fall indiziert ein positives $eq$, auch Anisotropieparameter genannt, eine längliche Form, die an eine Zigarre erinnert -- ein negatives indiziert eine flache Scheiben-Form.

Bei den -- in der NMR häufig verwendeten -- Magnetfeldern mit Stärken von mehreren Tesla gilt die Hochfeldnäherung, das bedeutet, dass die Zeeman-Wechselwirkung deutlich stärker ist als andere Wechselwirkungen, welche dann im Rahmen der Störungsrechnung untersucht werden dürfen. So ergibt sich für die erste Ordnung der Quadrupol-Wechselwirkung nach \cite[S. 209]{levitt}
\begin{align}
	\hat{H}_Q                   & = \omega_Q^{(1)} \frac{1}{6} \left( 3 {I_z}^2 - I(I + 1) \right) \\
	\text{mit}\; \omega_Q^{(1)} & = \frac{3e^2qQ}{2I(2I - 1) \hbar}.
\end{align}
Anstatt der Kopplungsfrequenz $\omega_Q$ wird auch häufig die Kern\-quad\-ru\-pol-Kopp\-lungs\-kon\-stan\-te $C_Q = e^2qQ / h$ in Hertz angegeben.

Es ist anzumerken, dass in der ersten Ordnung $H_Q$ nur vom Quadrat von $m$ abhängt und der Zentrallinienübergang von $-\sfrac{1}{2} \to \sfrac{1}{2}$ daher keine Korrektur erfährt. Dies ändert sich mit Betrachtung der zweiten Ordnung, was bei einer großen Stärke der Quadrupol-Wechselwirkung geschehen muss.

\par\bigskip

Da bei im Rahmen dieser Arbeit angestellten Messungen die zweite Ordnung von besonderem Interesse ist, wird im Folgenden eben darauf eingegangen -- ebenso wie auf daraus resultiertende Eigenschaften für messbare Größen wie $T_1$-Relaxation oder die Halbwertsbreite und die Verschiebung des Schwerpunkts von Spektren.

Es wird angenommen, dass ein sich im thermischen Gleichgewicht befindendes Spinsystem mit einem 90$^\circ$-Puls aus diesem ausgelenkt wird. Für die longitudinale Relaxation sagt die Theorie nach \cite{hubbard} im non-extreme narrowing Bereich, also bei $\omega \tau_c \gg 1$, eine biexponentielle Relaxation vorher:
\begin{align}
    - \frac{\Braket{I_z} - \Braket{I_z(\infty)}}{2\Braket{I_z(\infty)}} = \frac{4}{5}e^{-t/T_{1a}} + \frac{1}{5}e^{-t/T_{1b}}.
\end{align}
Dabei sind $T_{1a}$ und $T_{1b}$ gegeben als
\begin{align}
    1/T_{1a} &= 2J(2\omega_0) \\
    1/T_{1b} &= 2J(\omega_0).
\end{align}
Im extreme narrowing limit, $\omega \tau_c \ll 1$, verschmelzen die Exponentialfunktionen zu einer, da dort $J(2\omega_0) \approx J(\omega_0)$. Eine Spektraldichte $J \sim \tau_c / (1 + \omega^2 {\tau_c}^2)$ vereinfacht sich im Fall $\omega \tau_c \ll 1$ zu $J \sim \tau_c$, wodurch die Frequenzabhängigkeit verschwindet.

In der Praxis unterstützen die Daten laut \cite{eckert} allerdings auch sonst eine biexponentielle Relaxation selten, stattdessen ist eine monoexponentielle Funktion wie aus der bekannten BPP-Theorie \cite{bpp} eine hinreichende Näherung. Dort ergibt sich
\begin{align}
    1/T_1 &= K (J(\omega_0) + 4J(2\omega_0)) \label{eqn:bpp}
\end{align}
als einziger $T_1$-Wert, mit der kernabhängigen Konstante $K$.

Die Spektraldichte $J(\omega)$ kann je nach Modell unterschiedlich sein. In diesem Kontext wird eine Spektraldichte verwendet, die auf einer einzelnen Korrelationszeit $\tau_c$ für die Beschreibung der Molekülbewegungen basiert \cite{eckert}:
\begin{align}
    J_\text{BPP}(\omega) &= \frac{\pi^2}{5} {C_Q}^2 \left( 1 + \frac{\eta^2}{3} \right) \frac{\tau_c}{1 + \omega^2 {\tau_c}^2}. \label{eqn:spektraldichte_j}
\end{align}
Dabei ist $C_Q$ die Kopplungskonstante der Quadrupol-Wechselwirkung und $\eta$ der Asymmetrieparameter der Beschreibung des elektrischen Feldgradienten. Für $\tau_c$ wird oft ein Arrhenius-Gesetz mit der Aktivierungsenergie $E_a$ und dem präexponentiellen Faktor $\tau_{co}$ angenommen:
\begin{align}
    \tau_c = \tau_{co} \exp \left( \frac{E_a}{k_B T} \right).
\end{align}


Für die transversale Relaxation -- wieder nach einem 90$^\circ$-Puls -- gibt die Theorie \cite{werbelow}
\begin{align}
    \frac{\Braket{I_x} + \Braket{I_y}}{i\Braket{I_z}} = \frac{2}{5}e^{-i(\omega_0 + \omega_c^{(2)})t}e^{-t/T_{2c}} + \frac{3}{5}e^{-i(\omega_0 + \omega_s^{(2)})t}e^{-t/T_{2s}}. \label{eqn:trans_relax}
\end{align}
Auch hier sind zwei Exponentialfunktionen vorhanden (deren Effekt auch in Daten beobachtet werden kann); sie lassen sich als zugehörig zum Zentralübergang (Index $c$) bzw. zu den Satellitenübergängen (Index $s$) identifizieren. Dies kann, nach \cite{werbelow}, auch das Intensitätsverhältnis $\sfrac{2}{5}:\sfrac{3}{5}$ zwischen Zentralübergang und Satellitenübergängen wie folgt erklären: Die longitudinale Magnetisierung $\Braket{I_z}$ lässt sich durch die Populationen $P_i$ der jeweiligen Zustände $i$ ausdrücken als $\Braket{I_z} = (3/2)P_{3/2} + (1/2)P_{1/2} - (1/2)P_{-1/2} - (3/2)P_{-3/2}$. Einfaches Umformen ergibt $\Braket{I_z} = (3/2)(P_{3/2} - P_{1/2}) + (4/2)(P_{1/2} - P_{-1/2}) + (3/2)(P_{-1/2} - P_{-3/2})$. Hier lassen sich erster und dritter Term den Satellitenübergängen zuordnen, der zweite dem Zentralübergang; in der Summe ergibt sich dann genau das Verhältnis $\sfrac{4}{2}:\sfrac{6}{2}$ oder $\sfrac{2}{5}:\sfrac{3}{5}$.

$\omega_{c/s}$ sind die Schwerpunkte des Spektrums, die sich neben der Breite aus der transversalen Relaxation gewinnen lassen. Diese sind mit dem Imaginärteil der Spektraldichte, der dynamischen Frequenzverschiebung zweiter Ordnung, $Q(\omega, \tau_c)$, verbunden \cite{eckert}:
\begin{align}
    \omega_c^{(2)} &= Q(2\omega_0) - Q(\omega_0) \\ \label{eqn:schwerpunkt}
    \omega_s^{(2)} &= Q(\omega_0), \\
    \text{wobei } Q(\omega) &= \frac{\pi^2}{5} {C_Q}^2 \left( 1 + \frac{\eta^2}{3} \right) \frac{\omega {\tau_c}^2}{1 + \omega^2 {\tau_c}^2}.
\end{align}

Für die entsprechenden $T_2$-Werte $T_{2c}$ und $T_{2s}$ ergeben sich folgende Zusammenhänge \cite{eckert}:
\begin{align}
    1/T_{2c} &= J(\omega_0) + J(2\omega_0) \label{eqn:theo:T_2_dyn} \\
    1/T_{2s} &= J(0) + J(\omega_0).
\end{align}

Wird für eine solche Messung ein FID (free induction decay) aufgenommen, so lässt dieser sich mittels einer Fouriertransformation in ein Frequenzspektrum überführen. Bei der hier relevanten Exponentialfunktion entsteht eine Lorentzfunktion, deren Breite oder FWHM (full width at half maximum) $\Delta_{c/s}$ wie folgt mit $T_{2c}$ bzw. $T_{2s}$ verknüpft ist \cite{werbelow}:
\begin{align}
    \Delta_c &= 1/(\pi T_{2c}) \\ \label{eqn:fwhm}
    \Delta_s &= 1/(\pi T_{2s}).
\end{align}

Neben der hier vorgestellten Spektraldichte $J_\text{BPP}(\omega)$, die von einer einzelnen Korrelationszeit $\tau_c$ zur Beschreibung der Molekülbewegung ausgeht, existieren noch weitere Modelle, denen verschiedene Verteilungen von $\tau_c$ zugrunde liegen. Bekannte Spektraldichten sind von Cole-Cole ($J_\text{CC}$) und von Cole-Davidson ($J_\text{CD}$) \cite[S. 105-108]{beckmann_relaxation}, angepasst mit dem entsprechenden Vorfaktor aus Formel \eqref{eqn:spektraldichte_j}:
\begin{align}
\begin{split}
    J_\text{CC}(\omega, \tau_c, \alpha) &= \frac{\pi^2}{5} {C_Q}^2 \left( 1 + \frac{\eta^2}{3} \right) \frac{1}{\omega} \sin \left( \frac{\alpha \pi}{2} \right) \\ &\times \left[ \frac{(\omega \tau_c)^\alpha}{1 + (\omega \tau_c)^{2\alpha} + 2 (\omega \tau_c)^\alpha \cos (\alpha \pi / 2)} \right],
\end{split} \\
    J_\text{CD}(\omega, \tau_c, \gamma) &= \frac{\pi^2}{5} {C_Q}^2 \left( 1 + \frac{\eta^2}{3} \right) \frac{1}{\omega} \left\{ \frac{\sin [\gamma \arctan(\omega \tau_c)]}{(1 + \omega^2 {\tau_c}^2)^{2\gamma}} \right\}
\end{align}
Für $\alpha = 1$ bzw. $\gamma = 1$ korrespondieren diese Spektraldichten zu BPP.





\section{Pulsfolgen} \label{section:theo:pulsfolgen}

Um Größen wie $T_1$ und $T_2$ zu messen, werden unterschiedliche Pulsfolgen, also Sequenzen von RF-Pulsen mit bestimmten Längen und Abständen zwischeneinander, ausgenutzt.

Um die longitudinale Relaxation $T_1$ zu messen wurde die in Abbildung \ref{fig:theo:pulsT1} gezeigte Pulsfolge verwendet.
\begin{figure}
	\begin{center}
		\input{graphics/pulsfolgen/pulsfolgeT1.pdf_tex}
	\end{center}
    \caption{Pulsfolge zum Messen von $T_1$; sie besteht aus einem $\SI{180}{\degree}$-Invertierungspuls, gefolgt von einer Wartezeit $t_w$, einem $\SI{90}{\degree}$-Puls und einem Hahn-Echo mit der Evolutionszeit $t_p$. $t_w$ ist der variable Parameter. Abbildung aus \cite{joachim_master}}\label{fig:theo:pulsT1}
\end{figure}

Der erste Puls, ein $\SI{180}{\degree}$-Puls, invertiert die Magnetisierung von dem Gleichgewichtszustand $M_\infty$ zu $-M_\infty$, wonach sie anfängt zu relaxieren. Da die longitudinale Magnetisierung entlang $M_z$ aufgrund der Spulengeometrie nicht detektierbar ist, muss sie mit einem $\SI{90}{\degree}$-Puls in die Detektionsebene geklappt werden. Wird dies nach der Wartezeit $t_w$ getan, kann also die relaxierende Magnetisierung in Abhängigkeit von $t_w$ bestimmt werden und so Aufschluss über $T_1$ gewonnen werden.

Der letzte Puls, ein $\SI{180}{\degree}$-Puls nach der Evolutionszeit $t_p$, wird Hahn-Echo genannt. Häufig liegen die Momente direkt nach einem Puls in der Totzeit des Detektors, sodass die Magnetisierung faktisch erst gemessen werden kann, wenn sie eine gewisse Zeit vergangen ist, in der aber auch die Magnetisierung abklingt. Um dieses Problem zu umgehen wird das Hahn-Echo verwendet. Mit diesem können dephasierende Anteile der Magnetisierung, welche linear in $I_z$ sind, refokussiert werden; es kann dann ein „Echo“ des originalen Pulses zur Zeit $2 t_p$ beobachtet werden. Zu den betroffenen Wechselwirkungen zählen zum Beispiel lokale Feldinhomogenitäten \cite[S.302]{levitt}. Nicht jedoch wiedergewonnen werden können Teile der Magnetisierung, die durch $T_2$-Effekte verloren wurden. Damit dieser Einfluss bei den Messungen konstant ist, wird in diesem Fall das gleiche $t_p$ für alle Messungen verwendet.

Es lässt sich so aber offensichtlich mit der in Abbildung \ref{fig:theo:pulsT2} dargestellten Pulsfolge $T_2$ bestimmen:
\begin{figure}
	\begin{center}
		\input{graphics/pulsfolgen/pulsfolgeT2.pdf_tex}
	\end{center}
	\caption{Pulsfolge zum Messen von $T_2$; sie besteht aus einem $\SI{90}{\degree}$-Puls, gefolgt von einem Hahn-Echo. $t_p$ ist der variable Parameter. Abbildung aus \cite{joachim_master}}\label{fig:theo:pulsT2}
\end{figure}
Ein $\SI{90}{\degree}$-Puls kippt die Magnetisierung aus dem Gleichgewicht in die Detektionsebene, wo sie sodann anfängt zu dephasieren. Nach der Zeit $t_p$ wird ein Hahn-Echo durchgeführt und nach $2 t_p$ kann die verbliebene Stärke der Magnetisierung gemessen werden. $T_2$ kann so in Abhängigkeit von $t_p$ aufgenommen werden.

Ein einfacher $\SI{90}{\degree}$-Puls, ohne nachfolgendes Echo, wird FID -- kurz für free induction decay -- genannt. Er kann verwendet werden, wenn Einflüsse der Totzeit zu vernachlässigen sind -- neben experimentellen Umgebungen auch bei Simulationen.

Zum Studieren von Dynamik wird unter anderem die als stimuliertes Echo oder $F_2$ bezeichnete Pulsfolge verwendet, zu sehen in Abbildung \ref{fig:theo:pulsF2}:
\begin{figure}
	\begin{center}
		\input{graphics/pulsfolgen/pulsfolge2d.pdf_tex}
	\end{center}
	\caption{Pulsfolge zum Messen von stimulierten Echos; sie besteht aus einem $\SI{90}{\degree}$-Puls, gefolgt von einer Evolutionszeit $t_p$, einem $\pm \SI{90}{\degree}$-Puls, der Mischzeit $t_m$, einem $\mp \SI{90}{\degree}$-Puls und einer zweiten Evolutionszeit $t_a$. Um das Echo zu maximieren wird $t_a = t_p$ gesetzt. $t_m$ ist der variable Parameter. Abbildung aus \cite{joachim_master}}\label{fig:theo:pulsF2}
\end{figure}

Nach dem ersten Puls beginnt die Magnetisierung zu dephasieren. Der zweite Puls speichert -- je nach Pulssequenz -- entweder den Sinus- oder Kosinus-Anteil der aktuellen Magnetisierung als langlebige Längsmagnetisierung, während der jeweils andere Anteil mit $T_2$ zerfällt. Dementsprechend gibt es zwei Variationen der Pulsfolge, die als Sin-Sin oder Cos-Cos gekennzeichnet werden. Mit dem dritten Puls wird die gespeicherte Magnetisierung wieder in die Detektionsebene geklappt. 

Die resultierende Magnetisierung hängt dabei davon ab, ob die während der ersten Evolutionszeit $t_p$ akkumulierten Phasen mit denen der zweiten Evolutionszeit $t_a$ korreliert sind. Insbesonderen, wenn während der Mischzeit $t_m$ Dynamik stattfindet, also Atome beispielsweise diffundieren oder durch einen Sprung ihren Platz mit einem anderen Atom wechseln, ist dies nicht der Fall, was in unterschiedlichen Phasen und einem entsprechenden Abfall der Magnetisierung resultiert. Die Magnetisierung in Abhängigkeit von $t_m$ kann also als Maß für stattfindende Dynamik gesehen werden.

Findet während der Mischzeit keinerlei Dynamik statt, reduziert sich die Magnetisierung lediglich durch $T_1$, welche auch auf die Längsmagnetisierung wirkt. Daher macht es Sinn, das stimulierte Echo mit einer $T_1$-Messung zu vergleichen, um mögliche Differenzen festzustellen.

Damit diese Pulssequenz möglichst gut funktionieren kann, muss $t_p$ kurz genug sein, damit während dieser Zeit näherungsweise keine Dynamik stattfindet, $T_1$ lang genug, dass der Einfluss möglichst gering ist oder gar vernachlässigt werden kann, und $T_2$ kurz genug, dass die nach dem zweiten Puls verbleibende Magnetisierung schnell zerfällt.

Eine theoretisch fundiertere Darstellung kann \cite[insb. Kap. 6.2 und 10.2.3]{schmidt-rohr_multidimensional_1994} entnommen werden.

Es ist anzumerken, dass bei den meisten Pulsen nicht spezifiziert wurde, um welche Achse sie ausgeführt werden müssen. Dies liegt daran, dass es für die meisten Pulse mehrere Optionen gibt: Egal ob die Gleichgewichtsmagnetisierung mit einem $\SI{180}{\degree}$-Puls oder einem $\SI{-180}{\degree}$-Puls invertiert wird, ob um die $x$-Achse oder die $y$-Achse -- das Ergebnis ist immer die invertierte Magnetisierung $-M_\infty$. In der Praxis wird daher eine Kombination aus mehreren verschiedenen Pulsfolgen verwendet, die aber alle das gleiche Ergebnis liefern. Werden die Ergebnisse gemittelt, verstärken sich die Signale und unerwünschte Anteile -- beispielsweise durch leicht falsch gewählte Pulslängen -- mitteln sich heraus. So können weitere Fehlerquellen eliminiert werden.


Sowohl an $T_1$, $T_2$-Daten, als auch an Daten von stimulierten Echos, können Varianten einer Kohlrauschfunktion gefittet werden. Dabei handelt es sich um eine Exponentialfunktion, oder Kombinationen von Exponentialfunktionen, die mit einem Exponenten $\beta$ gestreckt werden können:
\begin{align}
	M_{T_1} (t_w) & = M_0 \left[ 1 - 2 \exp{ \left(- { \left( \frac{t_w}{T_1} \right) }^{\beta_{T_1}} \right)} \right] + M_\text{off} \label{eqn:theo:T_1_fit}                                                                     \\
	M_{T_2} (t_p) & = M_0 \exp{ \left(- { \left( \frac{2 t_p}{T_2} \right) }^{\beta_{T_2}} \right)} + M_\text{off} \label{eqn:theo:T_2_fit}                                                                                        \\
	M_{F_2} (t_m) & = M_0 \left[  \left(1 - Z\right) \exp\left(- \left(\frac{t_m}{\tau}\right)^{\beta_C}\right) + Z \right] \exp\left(- \left(\frac{t_m}{T_1}\right)^{\beta_{T_1}}\right) + M_\text{off}. \label{eqn:theo:F_2_fit}
\end{align}



\section{Der Glasbildner CRN} \label{section:theo:crn}

CRN, kurz für Calciumrubidiumnitrat, ist eine Mischung aus Calciumnitrat und Rubidiumnitrat. Der Schmelzpunkt hat bei einem Mischungsverhältnis von $2 : 3$ ein Minimum. Dieses Mischungsverhältnis liegt bei dem in dieser Arbeit verwendeten Stoff vor; dessen Summenformel lautet dementsprechend $[\text{Ca}(\text{NO}_\text{3})_\text{2}]_\text{2}[\text{RbNO}_\text{3}]_\text{3}$. Wird die Substand von einem geschmolzenen Zustand schnell genug abgekühlt, bildet sich ein Glas, dessen Glasübergangstemperatur bei $\SI{333}{K}$ liegt \cite{PIMENOV199793}. In der Umgebung dieses Punktes wurde ein Großteil der Messungen durchgeführt.

Es wurde an $^\text{87}$Rb gemessen, welches aufgrund seines höheren gyromagnetischen Verhältnisses von $\gamma = 2\pi \cdot \SI{13.984}{Mhz / T}$, trotz eines verhältnismäßig geringem natürlichen Vorkommen von $\SI{27.83}{\percent}$ dem häufigeren Isotop $^\text{85}$Rb für NMR-Untersuchungen vorzuziehen ist. $^\text{87}$Rb hat einen Kernspin von $\sfrac{3}{2}$ und ein Quadrupolmoment von $\SI{133.5}{mb}$.
