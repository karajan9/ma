Proben 

Bei den durchgeführten Messungen wurden drei verschiedene Proben verwendet.


Tabelle ***


Probe 1 stammt von Zürn ***; die Herstellung der Probe ist in *** dokumentiert. Proben 2 und 3 wurden für diese Arbeit angefertigt.
Dafür wurde CRN (*** was genau für welches?) mit Mörser und Stößel zu einem möglichst feinem Pulver zerrieben, welches in das entsprechende gerade oder gebogene (*** welches Material?)-Röhrchen gefüllt wurde. Um störende Einflüsse von Feuchtigkeit möglichst gut zu eliminieren, wurden die Proben über Nacht bei $\SI{160}{^\circ C}$ in einem evakuierten Ofen getrocknet, der im Anschluss mit Stickstoffgas gelöscht wurde.

So präpariert wurden die Proben unter Anwendung eines Freeze-Pump-Verfahrens (*** heißt das so?) von Herrn Kohlmann abgeschmolzen. Anschließend wurden die Röhrchen mit den Problem erhitzt um letztere zu schmelzen und sie dann durch Abkühlen in einen Glaszustand zu bringen

Mit dem gleichen Verfahren wurde am *** eine weitere Probe erstellt, welche aber vor dem Abschmelzen in den Glaszustand gebracht wurde. Größere Mengen aufsteigenden Gases könnten ein Indiz dafür sein, dass sich die Chemikalie zersetzt hat. Diese Probe wurde jedoch nicht verwendet, da Tests zeigten, dass sie kaum stabil im Glaszustand zu halten war sondern schnell Risse aufwies oder kristallisierte.
