Ofen 

NMR-Messungen bei mehr als $\SI{***}{^\circ C}$ durchzuführen kann sich mitunter als eine Herausforderung gestalten. Lötzinn, der häufig verwendet wird um die Resonanzspule an dem Probenkopf anzubringen schmilzt bei etwa $\SI{***}{^\circ C}$. Aber auch *** Tuning- und Matching- Spule/Kondensator *** sowie der für die Resonanzspule verwendete Draht sind nur für bestimmte Temperaturen ausgelegt. Zudem ist es offensichtlich, dass für höhere Temperaturen eine gute Isolierung und gegebenenfalls Kühlung des Probenkopfs gegeben sein muss, ist er doch in direkter Umgebung des mit flüssigem Helium gekühlten Magneten.

Für solche Anwendungen ist also ein spezieller Probenkopf notwendig, welcher in Form des Hochtemperaturprobenkopfs (intern als „Ofen“ bezeichnet) als eine Spezialanfertigung realisiert wurde.

Er lässt sich grob in drei Teilen beschreiben: Im oberen Drittel befindet sich der Schwingkreis mit der Probe, eine Vorrichtung, um diese zu erwärmen, sowie die dazugehörige notwendige thermische Isolation. Am unteren Drittel werden alle notwendigen Verbindungen angeschlossen; das mittlere Drittel verbindet beides und beherbergt einen Netzfilter.

Um Temperaturen von bis zu $\SI{1100}{K}$ erreichen zu können, wird ein Heizdraht aus einer FeCrAl-Legierung verwendet. Dieser ist bifilar über ein Hartkeramik-Hohlzylinder gewickelt, um möglichst wenig störende Magnetfelder entstehen zu lassen -- gerade in direkter Umgebung der Probe. Diese Vorrichtung ist im Deckel untergebracht und lässt sich so über die aus Platin bestehende Probenspule stülpen.

Um die entstehende Hitze möglichst gut nach außen hin zu isolieren, ist dieser Aufbau von poröser Keramik umgeben -- sowohl innerhalb des Deckels als auch zwischen Probenspule und Rest des Probenkopfes. Bei letzterem ist zudem eine weitere Scheibe aus dichter Keramik vorhanden. Für die Probenspule als auch Temperaturfühler sind Bohrungen in den Keramikscheiben vorhanden.

Darunter befindet sich der Schwingkreis der Probenspule. Sowohl eine Matchingspule, als auch ein variabler Kondensator und ein Halterung für einen wechselbaren Kondensator sind vorhanden. Es lassen sich Kondensatoren mit verschiedenen Kapazitäten in der Halterung anbringen; zusammen mit dem variablen Kondensator kann so für die Resonanzfrequenz eine Spanne von $\SI{63,6}{MHz}$ bis $\SI{168,9}{MHz}$ (mit Unterbrechungen) abgedeckt werden.

% Darunter befindet sich die Möglichkeit für eine Wasserkühlung, 

Der sich in dem Verbindungsteil befindliche Netzfilter soll den Heizstrom von störenden Anteilen befreien. Dazu wird eine Spule mit eiserner -- und damit magnetisierbarer -- Drossel (***) verwendet. Während diese Drossel den Probenkopf in der Vergangenheit genau im Hauptfeld des Magneten gehalten hatte, ist dies jetzt nicht mehr der Fall. Da zudem ungewünschte Einflüsse durch das zusätzliche Magnetfeld zu befürchten sind, liegt die Idee nahe, den vorliegenden Aufbau zu ändern. Eine mögliche Anpassung könnte sein, den Netzfilter außerhalb des Probenkopfes anzubringen und die Positionierung des Probenkopfes im Hauptfeld des Magneten durch eine zusätzliche Halterung zu gewährleisten.

Um den Aufbau zu betreiben sind im letzten Drittel eine Reihe von Anschlüssen vorhanden:
\item Ein BNC-Anschluss (***) der vom Richtkoppler kommend zum Schwingkreis führt
\item Anschlüsse (***) für die beiden Thermoelemente und
\item Ein Anschluss für den Heizstrom, die von einem Temperatur-Controller ausgelesen bzw. angesteuert werden
\item Je eine Rändelstange für die Einstellung von Matchingspule und Tuningkondensator
\item Anschlüsse für Zu- und Abflussschläuche für das Kühlwasser. Der Zufluss kann mit einem Wasserhahn gespeist werden.


„Reguläre“ Probenköpfe sind im Vergleich recht einfach gehalten. Da die Temperatursteuerung meist durch den Kryostaten geleistet wird, fallen die Heizeinheit, Isolation, Wasserkühlung und Netzfilter heraus und es bleiben nur Schwingkreis und Temperaturfühler. In dem hier verwendeten Probenkopf (***) ist zudem keine Matchingspule vorhanden, sodass sich das Abstimmen des Probenkopfen deutlich umständlicher gestaltet.
